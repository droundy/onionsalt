% copyright 2015 David Roundy daveroundy@gmail.com

% All rights reserved.  You have permission to distribute this paper
% without modification in either source code format, or in PDF format.

\documentclass[letterpaper,twocolumn,amsmath,amssymb,pre,aps,10pt]{revtex4-1}

\usepackage{graphicx}
\usepackage{listings}

\begin{document}

\title{The Onion Salt Layered Encryption Scheme}
\author{David Roundy}
\affiliation{Department of Physics, Oregon State University, Corvallis, OR 97331}

\begin{abstract}
  In this paper, I introduce a layered ``onion'' encryption scheme
  based on the NaCl library.  The onion salt protocol uses a slightly
  modified NaCl decryption for each layer, with a padding scheme that
  ensures that no intermediate agent can determine their own position
  in the sequence.  Nevertheless, each agent is able to ensure that
  the message was transmitted without modification of any routing
  information, although the payload intended for the recipient is only
  authenticated by the recipient.  The protocol allows for a
  round-trip onion with the recipient including a response to the
  message, with no router able to discern either who originated the
  messare or who was the recipient, and with each router only knowing
  the addresses of the routers before and after them in the sequence.
\end{abstract}

\maketitle

Onion routing was introduced in the nineties as a means to enable
anonymity over the network~\cite{reed1998onionrouting}.  Since then
Tor has moved from a system using self-contained onions, to a
telescoping set of channels~\cite{dingledine2004tor}.  Nevertheless, I
see value in a cryptographic primitive that enables the secure
construction of cryptographic onions to enable anonymizing routing of
short messages without the latency introduced by handshakes with each
intermediate router, and without that state be shared between the
routers and the sender.  By requiring multiple packets sent back and
forth, besides increasing latency, the use of sequential handshakes
introduces opportunities for traffic analysis attacks.  The use of
shared state introduces an additional per-connection cost to each
router, as well as the possibility of denial of service (DOS) attacks
in which an attacker opens numerous connections.

\newcommand\Nrouter{\ensuremath{N_{\text{router}}}}

\section{Properties of onion salt}

I introduce in this paper the \emph{onion salt} encryption scheme,
which uses the Networking and Cryptography library
(NaCl)~\cite{bernstein2009cryptography}, with one small modification,
for its grunt work.  This protocol creates an onion that is encrypted
to $\Nrouter$ ``routers,'' one of which is the ``recipient.''  The
onion contains plaintext designated for each party: routing
information for each router, and in addition a secret message
(``payload'') for the recipient.  The original onion can only be
decrypted by the first router, who thus obtains its routing
information and a message designated for the next router.  The
recipient in general is not the final receiver of the message, but
rather has the ability to attach a response payload to the message,
which can be received and read by originator of the onion, provided
the originator gives the final router his own address as the next
router.  The onion salt encryption scheme has the following
properties.
\begin{description}
\item[Secrecy] No eavesdropper without access to a given recipient's
  secret key may determine the plaintext intended for that recipient,
  even after intercepting every transmitted message.
\item[Authenticity] Each recipient can determine that the message
  received was not modified in any way, with the exception of the
  \emph{payload}, which is separately authenticated but only by the
  recipient (and the sender, in case of a response payload).
\item[Anonymity] No recipient can determine from the content of the
  message they received the identity of any other party involved,
  except for the routing information they may have been provided.  Of
  course, they most likely will also be able to identify who sent them
  the message by examining the transport used.
\item[Ordering ambiguity] No router can determine their place in
  the sequence of onion layers.  This last poses the challenge that
  motivates this work.  The previous three properties can be achieved
  simply by nesting NaCl encryption.  However, that would make each
  message larger than the previous one, making it easy for routers to
  establish a relative ordering between their messages.
\end{description}
In the following paragraphs I will explain the relevance of each of
these properties.

\emph{Secrecy} and \emph{authenticity} are at the core of any secure
communication scheme.  These are the two properties provided by the
NaCl \verb!crypto_box! function which we use.  However, we actually
use a modified version, which does \emph{not} authenticate the payload
content until its reception by its recipient.  This is needed in order
to enable the recipient to place information in the payload to be
relayed on.

\begin{figure*}
  \begin{center}
    \includegraphics[width=\textwidth]{decryption-0}
  \end{center}
  \caption{A diagram of the decryption process removing one layer of
    the onion---here the first layer.  Blocks of memory are
    represented by rectangles, and as those blocks are encrypted they
    are filled with colored hash lines corresponding to each
    encryption applied, which are removed with each decryption.  In
    step 1, the sender's public key is extracted, and the message is
    padded with zeros on the left and before the payload.  In step 2,
    the message is decrypted, which at the same time encrypts the zero
    padding in the middle.  Finally, once the routing information $R0$
    has been read, the message is truncated to the same size as the
    original message, and is ready to be passed
    on.}\label{fig:decryption}
\end{figure*}

The authenticity property as we implement it eliminates any potential
attackes that involve cleverly modifying an intercepted message
(e.g. flipping a few specific bits) and observing the resulting
behavior of its recipient, and ultimately compromising the secrecy of
the message.  A simpler approach would separately authenticate to each
router just the routing information for that router.  However, this
would leave open an attack in which one compromised router
communicates with a later compromised router in the sequence by
modifying a few bits of its routing information, enabling it to
identify packets that match.  By having each router authenticate
\emph{all} the routing information, an intermediate uncompromised
router would refuse to relay on such a modified packet.

The \emph{anonymity} of the sender is primarily ensured by generating
random key pairs for each recipient and transmitting the generated
public key along with the message.  Because each key pair is used only
once, although an eavesdropper can extract the public key from each
message, that information cannot be correlated with the sender.  In
addition, because each key pair is only used once, a fixed nonce may
be safely used.  Naturally, the secrecy of each router's message is
also necessary in order to ensure the sender's anonymity.  Also note
that my definition of anonymity explicitly made no claim regarding
traffic analysis attacks.  Naturally, if a user's threat model
includes an observer of network patterns (which might, for instance,
be performed by compromised routers), then additional care care---such
as batching or random delays in transmission---should be used to
protect against traffic analysis attacks.

Note that the sender may choose whether to remain anonymous to the
recipient.  To remain anonymous, the sender would generate a random
key pair for the payload, while to authenticate her identity, the
sender would encrypt the payload itself using her known public key.

Finally, \emph{ordering ambiguity} while not strictly necessary,
provides an additional measure of protection of anonymity in the
presence of malicious routers.  Simple padding with random data would
provide a limited ordering ambiguity that could prevent a network
observer from determining ordering.  However, if each router knew how
much authenticated data remained, traffic analysis by compromised
nodes could operate far more effectively when the ordering of messages
can be determined or estimated, easily eliminating half of the false
positives.  In the extreme case---where latency and ordering are
precisely known---a compromise of just a few routers in the sequence
could effectively remove the anonymity of a message.  This property
presents a challenge, because each router must be able to believe that
the entire routing portion of the onion consists of actual
uncompromised data.  Our solution is that each router must pad the
data it sends on with pseudorandom padding that is known to the
original sender, but not knowable to any other party.  Then the sender
can create authentication data for each recipient that authenticates
not only the actual important content, but also the padding.

\section{Decryption as a router}

I will begin with the decryption algorithm for an ordinary router,
illustrated in Fig.~\ref{fig:decryption}, which is simpler than
encryption.  Decryption consists of just three steps.
\begin{enumerate}
\item First the randomly-generated public key (which occupies the
  first 32 bytes) is read and removed from the message.  The message
  is then padded at the beginning with 16 bytes of zeros (as required
  by NaCl), and is zero-padded just before the payload with 48 bytes
  plus the size of the routing information.
\item The padded message is passed to a modified version of the NaCl
  function \verb!crypto_box_open! with a zero nonce, which decrypts
  the entire message, and authenticates all but the final payload
  portion of the message, in the process encrypting the zero padding
  that was inserted in the middle.  We note that in practice we modify
  the TweetNaCl implementation of NaCl for
  simplicity~\cite{bernstein2014tweetnacl}.
\item Finally, the decrypted routing information is read and stripped
  off, leaving a message of the same size as the original, padded with
  pseudorandom data in the center, and with the next
  randomly-generated public key at the beginning.
\end{enumerate}
One unusual feature of this scheme is the the \emph{ciphertext} is
padded with zeros in its interior prior to decryption and
authentication.  Also unusual is that our authentication does not
apply to the final portion of the message.

\begin{figure*}
  \begin{center}
    \includegraphics[width=\textwidth]{encryption}
  \end{center}
  \caption{A diagram of the encryption process, for a three-layer
    onion.  The steps are labelled by numbers in circles along the
    left-hand side. Blocks of memory are represented by rectangles,
    and as those blocks are encrypted they are given nested colored
    layers corresponding to each encryption applied.  Steps 0-4
    construct the padding needed for the innermost encryption, the
    core of the onion.  Step 5 inserts the secret information into the
    core.  Steps 6-10 encrypt that content in layers, along with
    routing information (the addresses $a_i$) at each outer layer of
    the onion.  In addiiton, session public keys $P_i$ are included
    along with the ciphertext sent to each recipient.  Finally, I
    note the authentication data $A_i$ at each level, which is the
    final contribution to the space overhead introduced with each
    layer.}\label{fig:algorithm}
\end{figure*}

\section{The encryption algorithm}

The first step in creating an onion salt is to generate one ephemeral
key pair for each recipient.  These keys will be used for each
encryption.  A cartoon of the process is shown in
Fig.~\ref{fig:algorithm}.  The encryption process is naturally more
complicated than decryption, since it must deal with all layers.  The
encryption proceeds in two stages.  First, the padding is constructed,
and then the plaintext for all the recipients is inserted and
encrypted.  The first stage consists of the following steps, starting
with a message full of zeros.
\begin{enumerate}
\item Shift the routing information to the left by 48 bytes plus the
  size of the routing information, and zero-fill the gap that was thus
  opened up.
\item Encrypt the resulting plaintext to the next recipient.
\item Repeat this process once for each router, including the
  recipient.
\end{enumerate}
Once we have generated the padding, we need to insert the plaintext
content (the secret message payload and the routing information) and
encrypt.  This process involves:
\begin{enumerate}
\item Insert the plaintext routing information for the current router.
  If this is the recipient, then also insert the payload, \emph{after}
  encrypting it to them with a user-specified key pair, using the
  randomly-generated emphemeral public key as a nonce.  The encryption
  of the payload uses a potentially different key pair (which is only
  ephemeral if the message is to be anonymous to the recipient), and
  uses the randomly-generated and ephemeral public key used for
  routing as the nonce.  This is intended to statistically ensure the
  uniqueness of the nonce in case the keys used are static, to avoid
  reusing a nonce.  GOT TO HERE!
\item The message is then encrypted to the recipient of the
  currently-constructed message.  If no mistakes have been made, this
  encryption will result in zeros at the end of the routing
  information.
\item Shift the routing information to the right to eliminate the gap
  of zeros, and then add the plaintext content consisting of the
  routing information for the next recipient and the ephemeral public
  key that recipient will use when decrypting.
\item Return to step 1, if there is another router.
\end{enumerate}
As in the case of decryption, the NaCl \texttt{crypto\_box} encryption
routine (even modified to not authenticate the payload portion of the
message) requires zero padding at the beginning of the message as well
as in the middle, introducing a bit more bookkeeping.

\begin{figure*}
  \begin{center}
    Router 1\\
    \includegraphics[width=\textwidth]{decryption-1}
    Recipient\\
    \includegraphics[width=\textwidth]{decryption-2}
    Router 3\\
    \includegraphics[width=\textwidth]{decryption-3}
    Router 4\\
    \includegraphics[width=\textwidth]{decryption-4}
    Router 5\\
    \includegraphics[width=\textwidth]{decryption-5}
  \end{center}
  \caption{A diagram of the decryption and responding
    process.}\label{fig:decryption-and-responding}
\end{figure*}

\begin{figure*}
  \begin{center}
    \includegraphics[width=\textwidth]{return-key}
  \end{center}
  \caption{Finding the return key.}\label{fig:return-key}
\end{figure*}

\section{Decryption and responding as recipient}

The purpose of \emph{not} authenticating the payload at each layer of
the onion is to enable the recipient to send a response without giving
the recipient any information about the sender beyond what is included
in the payload.  Specifically, the recipient cannot determine who the
sender is.  Moreover, no party other than the recipient (which
includes the other routers, and any network observer) can determine
which router is the recipient, provided the routers incorporate a
fixed (or randomized) delay in their forwarding on of the message,
such that the recipient has time to read the payload and encrypt the
response before the packet would otherwise be expected to be sent to
the following router.

Figure~\ref{fig:decryption-and-responding} shows a cartoon of the
entire sequence of messages passed between all routers following the
first one, which was already shown in Fig.~\ref{fig:decryption}.  Note
that the colored patterns representing encryption with various keys is
consistend across
Figs.~\ref{fig:decryption}-\ref{fig:decryption-and-responding},
representing the encryption of a single onion, followed by transmision
and decryption through the circuit.

The recipient has a somewhat more convoluted process than would
otherwise be required to insert the payload, because it is assumed
that the recipient will not recognize her role as recipient until
decrypting and reading the routing information, which may be presumed
to include a flag indicating that she is the recipient, and should
therefore examine the payload.  An alternative approach that would be
more resistent to timing attacks---but less resistent to
denial-of-service (DOS) attacks---would be have each router attempt to
decrypt the payload to its secret key, and use the authentication of
that decryption to determine whether they are the recipient.  On the
whole, this process does not seem warranted, when the routers can
defeat timing attacks more efficiently by simply sleeping until a
determined time (e.g. a random interval after receiving the packet, or
a time specified in the routing information).

The reciever, once she has identified herself as such, can immediately
examine the payload, which presumably is itself encrypted, and
certainly must be authenticated.  Having done so, she constructs her
response, presumably also encrypted and authenticated.  We assume that
the response payload is encrypted and indistinguishable from random,
and we therefore simply overwrite the original payload.  This
encryption is not reflected in the figure, since it is not part of the
onion salt protocol itself.  One option would be for the recipient to
encrypt her response to the sender's ephemeral public key using NaCl
with a different nonce, either a random one, or perhaps a value of 1.

The subsequent routers (routers 3, 4, and 5 in our example) are
unaware that they are relaying a response rather than the original
payload.  We presume that the final router is given the sender's
address.  The sender will decrypt the response by encrypting it (or
decrypting without authentication) to each of the keys of the routers
following the recipient.  This will restore the response to what the
recipient inserted, which can then be decrypted and authenticated.

FIXME I NEED TO EDIT THIS TO REFLECT THE ENCYPTION AND AUTHENTICATION
OF THE PAYLOAD, INCLUDING NONCES.

\section{Extensions}

An obvious extension to this scheme is to enable different routers to
recieve and respond to different portions of the payload.  To achieve
this, the routing information would inform each router which
randomized portion of the payload to read and respond to. This could
enable collection of information from an entire circuit in one
message.  The procedure would be essentially identical to the
single-recipient case, and we will consider it no further in this
paper.

Another alternative is that the circuit need not return to the
original sender.  One could either use the protocol to send
information in just one direction, or the sender could cause the
recipient to relay information to a third party.  The latter would
require a second message to inform the third party of the shared
secrets needed to decrypt the ``response.''

\section{Analysis}

Most of the security properties claimed for onion salt derive from
those of the NaCl \texttt{crypto\_box} function (secrecy and
authentication), and from the generation of new random key pairs for
each recipient (anonymity).  The ordering ambiguity derives from the
padding scheme.  There are, however, a few features of this scheme
that bear examination lest they violate the preconditions needed for
NaCl to provide the desired secrecy and authentication.

One such feature is the use of a zero nonce.  I believe this is safe
because we only use each key pair to send a single message.

The primary risk that I see in this scheme is in whether the choice of
padding could introduce a flaw in the encryption properties.  In
particular, we encrypt twice with the same key pair and nonce, which
has the potential for danger when using stream ciphers.  However, the
only bytes that are encrypted twice which we use are those that were
originally zero.  The recipient of padded bytes generated using this
encryption gains the same knowledge about our stream cipher that an
opponent would gain using a known-plaintext attack.  Because I believe
the Salsa20 stream cipher is resistent to known-plaintext attacks, I
also believe that there should not be an exploit that can take
advantage of the nature of this padding.

Another question is that of timing attacks.  NaCl itself is
constructed to be resistant to such attacks, because branches never
depend on secret data, and array indexing never depends on secret
data.  A timing attack on the creation of an onion could certainly
reveal how many layers it has, and possibly some information about the
slowness of the random byte generator, but should not reveal any more
information.  Decryption of a single layer by a router should happen
in fixed time, independent of the data involved, and thus should be
resistant to timing attacks.  Of course, the timing of the actual
routing process will likely be very dependent on the secret content,
but that must be addressed at a different level of the protocol.

\section{Cost}

The cost to create an onion approximately twice the cost to encrypt an
equal-sized message to each recipient.  A more efficient
implementation would avoid generating the authentication codes while
preparing the padding, and would only generate the shared symmetric
key once.  These optimizations will not affect the resistence to
timing attacks, and can safely be left for a later improvement.

\section{Conclusions}

I have introduced a new encryption protocol, onion salt, which creates
a nested sequence of encryptions suitable for onion-style routing of
fixed-size packets without handshake or prior shared secrets.  This
protocol has the properties of secrecy, authentication of each layer,
anonymity, and order ambiguity.  The latter means that routers are
unable to determine their position in the sequence.  The cost of this
order ambiguity is in padding of data, as well as increasing the
amount of data that need be encrypted.  I believe this scheme could be
valuable for low-bandwidth anonymous communications.

%% Tarzan is a low-latency peer-to-peer anonymizing layer that acts at
%% the IP level~\cite{freedman2002tarzan}.  The second-generation Tor
%% router uses a telescaping connection-based encryption
%% scheme~\cite{dingledine2004tor} rather than the ``onion-based'' system
%% of the original onion routing~\cite{reed1998onionrouting}.  Note that
%% the latter paper (the old one) is actually a very nice
%% read~\cite{reed1998onionrouting}.  Aqua is an interesting and recent
%% high-bandwidth anonymizing network~\cite{leblond2013towards}.

%% Here is a nice paper to read on distributed hash tables and security
%% considerations involved~\cite{sit2002security}.  And here is a nice
%% one about how you need a critical mass in order to ensure
%% anonymity~\cite{dingledine2006anonymity}.

\bibliography{onionsalt}% Produces the bibliography via BibTeX.

\appendix

\begin{widetext}

\section{onionsalt.h}

\lstinputlisting[language=C]{../src/onionsalt.h}

\section{onionsalt.c}

\lstinputlisting[language=C]{../src/onionsalt.c}

\end{widetext}

\end{document}
