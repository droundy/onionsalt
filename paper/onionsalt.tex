\documentclass[letterpaper,twocolumn,amsmath,amssymb,pre,aps,10pt]{revtex4-1}

\usepackage{graphicx}

\begin{document}

\title{The Onion Salt Layered Encryption Scheme}
\author{David Roundy}
\affiliation{Department of Physics, Oregon State University, Corvallis, OR 97331}

\begin{abstract}
  In this paper, we introduce a layered ``onion'' encryption scheme
  based on the NaCl library.  The onion salt protocol uses unmodified
  NaCl decryption for each layer, with zero padding in the plain text
  to ensure that the agent decrypting a given layer has no means to
  determine either how many layers remain or the size of the interior
  message.  As in the standard NaCl unbox function, each agent
  removing a layer of encryption can authenticate that the message was
  not modified in transit.  In order to acheive this we do some clever
  tricks.
\end{abstract}

\maketitle

Tarzan is a low-latency peer-to-peer anonymizing layer that acts at
the IP level~\cite{freedman2002tarzan}.  The second-generation Tor
router uses a telescaping connection-based encryption
scheme~\cite{dingledine2004tor} rather than the ``onion-based'' system
of the original onion routing~\cite{reed1998onionrouting}.  Note that
the latter paper (the old one) is actually a very nice
read~\cite{reed1998onionrouting}.  Aqua is an interesting and recent
high-bandwidth anonymizing network~\cite{leblond2013towards}.

Here is a nice paper to read on distributed hash tables and security
considerations involved~\cite{sit2002security}.  And here is a nice
one about how you need a critical mass in order to ensure
anonymity~\cite{dingledine2006anonymity}.

\begin{figure*}
  \begin{center}
    \includegraphics[width=\textwidth]{onion-decryption}
  \end{center}
  \caption{A diagram of the decryption process.  Red dotted lines
    represent decryption, and black dotted lines simply represent
    cutting out the address information.}
\end{figure*}

\bibliography{onionsalt}% Produces the bibliography via BibTeX.

\end{document}
